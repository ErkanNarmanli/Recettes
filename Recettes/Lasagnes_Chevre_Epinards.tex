\begin{recette}{Lasagne chèvre épinards}
  \emph{Pour 4 personnes, Préparation 30 min, Cuisson 40 min environ}
  \begin{ingredients}
    L’équivalent volumique d’un sac plastique d’épinards\sep
    Un demi paquet de pâtes à lasagne\sep
    1 grosse bûche de chèvre\sep
    Un peu de crème fraîche\sep
    Un peu de béchamel (très peu muscadée)\sep
    Un peu d’emmental râpé\sep
  \end{ingredients}

  \recetteSection{Recette}
	\begin{enumerate}
	\item Lavez et faites cuire les épinards à la vapeur ou à la poêle avec un peu d’huile d’olive et un peu de crème.
  \item Pendant ce temps, coupez la bûche en rondelles. Montez la lasagne en commençant par de la crème fraîche, puis pâte, épinards, chèvre, crème, pâte etc. Finissez par une couche généreuse de béchamel (elle doit tout recouvrir pour pas que la dernière couche de pâte ne brûle) et un peu d'emmental râpé.
  \item Enfournez à 180 degrés. Testez la cuisson avec un couteau, vous ne devez pas sentir les pâtes à lasagne sous le couteau. Salez et poivrez à convenance. \\
  \item Vous pouvez aussi faire cette recette avec du Saint Marcellin bien fait, dans ce cas là, ajustez un peu les proportions en faveur des épinards.
  \end{enumerate}
\end{recette}
